\section{Problem Formulation}
\label{sec:notation}

% In this section, we provide background on uncertain graph, motivate the basic privacy attack, and formulate the uncertain graph anonymization problem. 
In this section, we provide background on uncertain graph, privacy attack and justify our choice of utility loss metric. 
On this basis, we present our formulation of the uncertain graph anonymization problem. 

\vspace{-4pt}
\subsection{Uncertain Graph}
\vspace{-2.5pt}
An uncertain graph $\mathcal{G}=(V,E,\mathit{p})$, is defined over a set of nodes $V$, a set of edges $E$, and a set of probabilities $\mathit{p}$ of edge existence. Following the literature~\cite{Potamias_K_2010,Zhao_Detecting_2014,Colbourn_Colbourn_1987}, we assume possible-worlds semantics, and we consider the edge probabilities independent~\footnote{We leave the conditional probability model as a future
extension.}. An uncertain graph $\mathcal{G}=(V,E,\mathit{p})$ essentially represents a probability distribution over all of the certain graphs $G$ in the forms of which the uncertain graph may actually exist. 
The probability of observing any possible world $G_i=(V,E_{G_i})$ is    
\begin{equation*}
    Pr[G_i]=\prod_{e \in E_{G_i}} {\mathit{p}(e)} \prod_{e \in E \setminus E_{G_i}} 1-\mathit{p}(e)
\end{equation*}

\subsection{Privacy Attack}
\label{sec:AMPC}
\vspace{-5pt}
\begin{figure}[!htb]
  \vspace{-10pt}
    \subfigure[Social Trust Network]{\label{fig:socialNetwork}
      \begin{minipage}[l]{0.46\columnwidth}
        \centering
        \includegraphics[height=2.7cm]{ill/example_source.pdf}
      \end{minipage}
      }
    \subfigure[The naive anonymization]{\label{fig:b2bNetwork}
      \begin{minipage}[l]{0.46\columnwidth}
        \centering
        \includegraphics[height=2.7cm]{ill/example_output.pdf}
      \end{minipage}
      }
    \vspace{-7pt}
    \caption{The Structural Re-Identification Issue.}
    \label{fig:privacyAttack}
    \vspace{-7pt}
\end{figure} 
Apparently, simply removing the identities of the nodes before publishing the uncertain graph does not guarantee privacy.  The structure of the uncertain graph itself, and in its basic form the degree of the nodes, can be revealing the identities of individuals. 
In practice, the adversary may have access to external information about the entities in the graphs. This information may be obtained by the adversary's malicious actions. 
For example, for the uncertain graph in Figure~\ref{fig:privacyAttack}, the adversary might know that ``Fred has three or more \textbf{trust} neighbors''. Such information allows the adversary to narrow down the set of candidates in the sanitized graphs. For example, the statement partially re-identify Fred as $\lbrace 2,3,6 \rbrace$ with \textbf{probabilities} respectively. 
Different to the deterministic scenario,  there are different posterior probabilities over candidate nodes $\lbrace 2,3,6 \rbrace$ where $P(6|\text{Fred}) \simeq P(3|\text{Fred}) \gg P(2|\text{Fred})$.

The node is vulnerable to the re-identification risk. Entity Re-identification (ER) can lead to additional disclosures. In this paper, we focus on the ER attack as this attack is one of the most serious privacy problems. 

\vspace{-12pt}
\subsection{Privacy Notion}
\label{sec:privacyNotion}
To resist re-identification attacks, we adopt the $(k,\epsilon)$-obf, an syntactic privacy notion,
% introduced by Boldi {\etal}~\cite{Boldi_Injecting_2012},
 where $k \ge 1$ is a desired level of obfuscation  and $\epsilon \ge 0$ is a tolerance parameter. 

\textsc{Obfuscation Parameter}~~Similar to $k-$anonymity, $k-$obf requires blending every entity with other fuzzy matching entities. While, the level of obfuscation is quantified as the entropy over posterior probabilities over fuzzy matching ones. It lower bounds the entropy of the distribution by $\log_{2} k$. 
\emph{Though it is initially used to measure the anonymity provided by an uncertain graph to the deterministic graph, the stochastic nature makes it a good fit in the uncertain scenario.} 

\textsc{Tolerance parameter}~~As for the tolerance parameter $\epsilon$, it serves for the following purpose. There might be extreme unique nodes, e.g., Trump in a Twitter network, whose obfuscation is almost impossible. Thus, Boldi {\etal}~\cite{Boldi_Injecting_2012} introduce a tolerance parameter $\epsilon$, which allows skipping up to $\epsilon * |V|$ nodes and makes the privacy goal more practical. 

% The formal definition is,
% \theoremstyle{definition}
% \begin{definition}
% 	\textbf{\boldmath{$(k,\epsilon)$}-obf \cite{Boldi_Injecting_2012}}
%     Let $P$ be a vertex property (i.e., vertex degree in our work), $k \geq 1$ be a desired level of anonymity, and $\epsilon >0 $ be a tolerance parameter. 
%     An sanitized uncertain graph $\tilde{\mathcal{G}}$ is said to $k$-obfuscate a given vertex $v \in \mathcal{G}$ w.r.t $P$ if the entropy $H()$ of the distribution $Y_{P(v)}$ over the nodes 
%     of $\tilde{\mathcal{G}}$ is greater than or equals to $\log_{2}{k}$:
%     \begin{equation*}
%         H(Y_{P(v)}) \geq \log_{2}{k}.
%     \label{obfCon}
%     \end{equation*}
% The uncertain graph $\tilde{\mathcal{G}}$
% is $(k,\epsilon)$-obf w.r.t property $P$ 
% if it $k$-obfuscates at least $(1-\epsilon)|V|$ nodes in $\mathcal{G}$. 
% \label{def:obf}
% \end{definition} 
\vspace{-12pt}
\subsection{Utility Loss Metric: Reliability Discrepancy}

As a fundamental property, connectivity plays a vital role in graph mining tasks such as nearest neighbor locating and clustering. 
The connectivity model can yield a better graph representation than the degree sequence model. 
Motivated by the above, connectivity discrepancy is widely used to measure the structural difference between deterministic graphs. 

The concept of reliability generalizes the connectivity concept uncertain scenario. 
It captures the probability that two given nodes are reachable over all possible worlds, as shown in Def~\ref{d:reliability}. 
Analogous to the deterministic case, we use reliability discrepancy as the utility-loss metric in the uncertain scenario, as outlined in Def~\ref{d:RD}. 
\begin{definition}
    \textbf{Two-Terminal Reliability~\cite{Colbourn_Colbourn_1987}}~~Given an uncertain graph $\mathcal{G}$, and two distinct nodes $u$ and $v$ in the graph, the reliability of $(u,v)$ is defined as:
        \begin{equation*}
                R_{u,v}(\mathcal{G})= \sum_{G \in W(\mathcal{G})} \mathcal{I}_{G}(u,v) ~ Pr[G] 
        \end{equation*}
    where $\mathcal{I}_{G}(u,v)$ is 1 iff $u$ and $v$ are contained in a connected component in $G$, and 0 otherwise.   
    \label{d:reliability}
\end{definition}

\theoremstyle{definition}
\begin{definition}
    \textbf{Reliability Discrepancy (RD)}
    The reliability difference between a sanitized output $\tilde{\mathcal{G}}$ and the original input $\mathcal{G}$, 
    denoted as $\Delta(\tilde{\mathcal{G}})$, 
    is defined as the sum of the two-terminal reliability discrepancy over all node pairs $(u,v) \in V_\mathcal{G}$.
    \begin{equation*}
        \Delta(\tilde{\mathcal{G}})=\sum_{(u,v) \in V_\mathcal{G} }|R_{u,v}(\mathcal{G})-R_{u,v}(\tilde{\mathcal{G}})|
    \end{equation*}
    \label{d:RD}
\end{definition}

\subsection{Problem Statement} 
% \vspace{-5pt}
\begin{problem}
     Given an uncertain graph $\mathcal{G}$ and desired anonymization parameters $k$ and $\epsilon$, 
     the objective is to find a  $(k,\epsilon)$-obf uncertain graph $\tilde{\mathcal{G}}$
     with the minimal utility loss,
     \vspace{-5pt}
     \begin{equation*}
             \begin{aligned}
                 & \argmin_{\tilde{
                \mathcal{G}}} & & \Delta(\tilde{\mathcal{G}}) \\
                &  \text{Subject to} & &\tilde{\mathcal{G}} \text{~is~} (k,\epsilon)-obf
            \end{aligned}
     \end{equation*}
     \label{prob:unobf}
\end{problem}
% add one paragraph to ....