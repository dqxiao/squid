% IEEE Paper Template for US-LETTER Page Size (V1)
% Sample Conference Paper using IEEE LaTeX style file for US-LETTER pagesize.
% Copyright (C) 2006-2008 Causal Productions Pty Ltd.
% Permission is granted to distribute and revise this file provided that
% this header remains intact.
%
% REVISION HISTORY
% 20080211 changed some space characters in the title-author block
%
\documentclass[10pt,conference,letterpaper]{IEEEtran}
\usepackage{times,amsmath,epsfig}

\usepackage{subfigure}
\usepackage{color}
\usepackage{balance}  % for  \balance command ON LAST PAGE 
\usepackage{graphicx}
\usepackage{epstopdf}  % used for generating pdf 
\usepackage[english]{babel}
\usepackage[utf8]{inputenc}
\usepackage{amsmath}
\usepackage{graphicx}
\usepackage{caption}

\let\proof\relax
\let\endproof\relax
\usepackage{amsthm}


\usepackage{caption}
\usepackage{color}
\usepackage[ruled]{algorithm}
\usepackage{algorithmic}
\renewcommand{\algorithmiccomment}[1]{\bgroup #1 \egroup}
\usepackage{cite} % to sort cite 
\usepackage{multirow} % to use multirow 


\newcommand{\ie}[0]{\textit{i.e.}}
\newcommand{\eg}[0]{\textit{e.g.}}
\newcommand{\etal}[0]{\textit{et al.}}

\newcommand{\keobf}{\textit{$(k,\epsilon)$-obfuscation}}
\newcommand{\argmin}{\operatornamewithlimits{argmin}}
\newcommand{\norm}[1]{\left\lVert#1\right\rVert}
\newcommand{\repAn}[0]{\textnormal{Rep-An}}
\newcommand{\funcname}[0]{\textit{GenAnonymization}}

\theoremstyle{plain}
\newtheorem{theorem}{Theorem} 
\newtheorem{definition}{Definition}
\newtheorem{problem}{Problem}
\newtheorem{example}{Example}
\newtheorem{lemma}{Lemma}

\newcommand{\SysName}{Chameleon }
\newcommand{\SysNameNS}{Chameleon}

%
\title{Chameleon: Uncertain Graph Releasing Using Synthetic Private Uncertain Graph Models}

% \author{Dongqing Xiao,  Mohamed Y. Eltabakh, Xiangnan Kong \\ 
% \affaddr{Worcester Polytechnic Institute (WPI), Computer Science Department, MA, USA} \\
% \{dxiao,~meltabakh,~xkong\}~@wpi.edu}  
\author{%
% author names are typeset in 11pt, which is the default size in the author block
{Dongqing Xiao, Xiangnan Kong, Mohamed Y. Eltabakh}%
% add some space between author names and affils
\vspace{1.4mm}\\
\fontsize{10}{10}\selectfont\itshape
Computer Science Department, Worcester Polytechnic Institute \\
Worcester, United States of America\\
\fontsize{9}{9}\selectfont\ttfamily\upshape
\{dxiao,~meltabakh,~xkong\}~@wpi.edu\
}
\begin{document}
\maketitle


% NOTE keywords are not used for conference papers so do not populate them
% \begin{keywords}
% keyword-1, keyword-2, keyword-3
% \end{keywords}
%

\begin{abstract}  
Research on soical and business applications requires open access to real datasets. While these dataset can be shared, generally in the form of uncertain graphs whose edges are labeled with a probability of existence, doing so often risks exposing sensitive user information to the public. Current techniques to improve privacy on graphs only target determinitic ones, vulnerable against powerful de-anonymization attacks over uncertain graphs. Our work a solution to release uncertain graphs with high utility while preserving privacy. 
\end{abstract}

\section{Introduction}

\label{sec:Intro}

% Graph-- Uncertain Graph--Examples 
Graphs are used to capture the complex relationships in emerging applications, such as business to business (B2B) and social networks. 
Sometimes, the existence of the relationship between two entities is uncertain. For instance, in social networks, nodes represents individual users, while edges represent friendship or trust link among them.  Usually, the link is derived by inference and prediction models built on interaction details~\cite{Lin_B2B,Adar_Managing_2007,Kempe_Maximizing_2003}. And, edge probability denotes the accuracy of a link prediction, or the trust of one person on another. 
In these applications, the data can be modeled and shared as uncertain graphs whose edges carries a probability of existence. The probability represents the confidence that the relationship holds in reality. 

\begin{figure}[!htb]
  \vspace{-7pt}
    \subfigure[Social Trust Network]{\label{fig:socialNetwork}
      \begin{minipage}[l]{0.46\columnwidth}
        \centering
        \includegraphics[height=2.7cm]{ill/SocialNetwork.pdf}
      \end{minipage}
      }
    \subfigure[B2B Network]{\label{fig:b2bNetwork}
      \begin{minipage}[l]{0.46\columnwidth}
        \centering
        \includegraphics[height=2.7cm]{ill/B2BNetwork.pdf}
      \end{minipage}
      }
    \vspace{-7pt}
    \caption{Real-world uncertain graphs with privacy concerns.}
    \label{fig:motivation}
    \vspace{-7pt}
\end{figure} 

% Sharing--Privacy--Examples 
These uncertain graphs are invaluable for scientific research and commercial applications~\cite{Kempe_Maximizing_2003,Cho_Friendship_2011}. However, sharing these uncertain graphs would violate the privacy of users or entities profiled inside. In social trust network, the trust relationships among users--- which greatly impact users' behaviors, are usually probabilistic.  They are useful in social interaction study and micro-targeting. While users are unwilling to share such confidential information with potential adversaries like Cambridge Analytica. In B2B networks, business operators also hesitate to share transaction patterns as it relates to  confidential business models. Such tension is raising the question of sharing uncertain graphs without compromising privacy. 


% State-of-Art 
A number of privacy preserving graph sharing schemes have been studied in the deterministic scenario~\cite{Liu_Towards_2008,Ying_Randomizing_2008,Wang2011,Liu_Privacy_2009,Nguyen_Anonymizing_2015,Sala_Sharing_2011,Xiao_Differentially_2014,lee2011}, though many problems about sharing uncertain graphs still remain unexplored.

%weight casting fails 
An obvious approach is to convert uncertain graph sharing problem into the deterministic case by casting edge probabilities as edge weights. However, by disregarding the possible world semantics of the uncertain graph, such an approach fails to reflect uncertain graph properties such as connectivity, dense subgraphs correctly~\cite{Zhao_Detecting_2014,Hua_Probabilistic_2010}. {\small Note that connectivity of deterministic subgraphs is generally measured by the concept of cut, which is defined as the sum of weights of intra edges. Generally, the bigger the cut, the harder to separate two subgraphs. In Figure~\ref{fig:socialNetwork}, the equal cut $C(SG_{1},SG_{2})=C(SG_{3},SG_{2})=1$ implies the identical connectivity of $SG_{1}$ and $SG_{3}$ w.r.t $SG_{2}$. However, with the possible world semantics, we know the probability to separate $SG_{1}$ and $SG_{2}$ is $(1-0.5)^{2}=0.25$, and that to separate $SG_{2}$ and $SG_{3}$ is $(1-0.3)(1-0.7)=0.21$. Hence, in fact, $SG_{2}$ is closer to $SG_{1}$ than to $SG_{3}$.} 
Hence, the weighted graph anonymization scheme could produce very poor result in the uncertain scenario even if the anonymization algorithm is good. 


% Rep-OB 
In previous work~\cite{Xiao:2018}, we ever present another option, Rep-An. 
It first extracts a single deterministic representative
instance $G$ that capture structural properties of the uncertain
graph.
After that, anonymization can be then be proceed efficiently on $G$ using conventional algorithms, regardless of the uncertainty.  
However, Rep-An is is not always feasible. The detachment of edge uncertainty deteriorates the data utility. 


As ever mentioned, either the casting or detachment of edge probabilities lead to poor results. Existing graph anonymization schemes are inadequate to share uncertain graphs with desirable trade-off between privacy and utility. 
There are the following distinct challenges in the uncertain scenario. 

$\bullet$~\textup{\emph{Stochastic Privacy Attacks.}}~~Edge uncertainty plays an indispensable role in the uncertain graph model. It is impractical to discard them in the release.  
While the extra release of edge uncertainty makes privacy protection far more difficult as it empowers the adversary and makes the profiled entity more vulnerable. 
% To this end, we show the potential re-identification attack and present the corresponding solution. 

$\bullet$~\textup{\emph{Stochastic Utility Loss Metric.}}~~It is very challenging to maintain the graph structure when the uncertain graph is modified to pursue anonymity. The structural distortion incurred is evaluated by the corresponding metric.  It plays the key role in utility preserving. Unfortunately, existing graph utility loss metrics such as graph edit distance~\cite{Liu_Towards_2008}, spectrum discrepancy~\cite{Ying_Randomizing_2008}, community reconstruction error~\cite{Wang2011} and shortest path discrepancy~\cite{Liu_Privacy_2009} are not suitable in the uncertain scenario because of the ignorance of edge probability. 
% In this context, the discrepancy w.r.t standard uncertain graph reliability becomes a good criterion. It evaluates the connectivity difference in the context of the entire graph and meanwhile utilizes the possible world model. 

$\bullet$~\textup{\emph{Intractable Search Space.}}~~It is very challenging to find a sanitized graph with the desired level of privacy by as few graph operations as possible. 
In the deterministic scenario, the problem is known to be NP-hard by edge additions and deletions~\cite{Hartung_Theory_2015}. 
In the uncertain scenario, the edge modification is no longer a binary operation (addition/deletion), but can be infinite probability values. Exhaustive search is computationally intractable if the number of edges is large.
 % Thus, we approximate the problem of interest via a randomized algorithm, which built on the basis of meta-heuristics. It excels in identifying a population of sanitized results with good quality.

In this work, we propose a solution tailored towards uncertain graphs via incorporating possible world semantics, {\methodName}. 
% {\methodName} is built on the basis of syntactic private notation. 
It preserves as much the stochastic nature of the original uncertain graph as possible, while injecting enough structural noise to guarantee a chosen level of privacy against re-identification attacks.
Specifically, we make the following contributions:
\begin{itemize}
\item To the best of our knowledge, we are the first to formulate the uncertain graph anonymization problem. 
 We show the potential re-identification attack and present the corresponding privacy notion. 
\item We propose an utility loss metric on the basis of generalized reliability. It evaluates the connectivity difference in the context of the entire graph and meanwhile utilizes the possible world model. 
\item We propose a randomized algorithm with dual meta-heuristics. By incorporating the possible world semantics, it excels in identifying a population of sanitized results with good quality.
\item We conduct extensive experimental studies to demonstrate efficiency and practical utility of our algorithms.
\end{itemize}

The rest of the paper is organized as follows. In Section 2, we summarize related works and clarify our distinct privacy goal. In Section 3 we formulate the uncertain graph-anonymization problem. Sections 4 – 5 consider the anonymization problem in the context of uncertain graphs.  In Section 6 we apply our method to several real-world uncertain graphs, and demonstrate their efficiency and practical utility. 


\bibliographystyle{abbrv}
\bibliography{refs}

\end{document}
