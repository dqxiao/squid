% IEEE Paper Template for US-LETTER Page Size (V1)
% Sample Conference Paper using IEEE LaTeX style file for US-LETTER pagesize.
% Copyright (C) 2006-2008 Causal Productions Pty Ltd.
% Permission is granted to distribute and revise this file provided that
% this header remains intact.
%
% REVISION HISTORY
% 20080211 changed some space characters in the title-author block
%
\documentclass[10pt,conference,letterpaper]{IEEEtran}
\usepackage{times,amsmath,epsfig}

\usepackage{subfigure}
\usepackage{color}
\usepackage{balance}  % for  \balance command ON LAST PAGE 
\usepackage{graphicx}
\usepackage{epstopdf}  % used for generating pdf 
\usepackage[english]{babel}
\usepackage[utf8]{inputenc}
\usepackage{amsmath}
\usepackage{graphicx}
\usepackage{caption}
% \usepackage[compress]{cite}
\usepackage[numbers,sort&compress]{natbib}
\renewcommand{\bibfont}{\footnotesize}
\usepackage{amsfonts} % for mathbb

\let\proof\relax
\let\endproof\relax
\usepackage{amsthm}


\usepackage{caption}
\usepackage{color}
\usepackage[ruled]{algorithm}
\usepackage{algorithmic}
\renewcommand{\algorithmiccomment}[1]{\bgroup #1 \egroup}




\newcommand{\ie}[0]{\textit{i.e.}}
\newcommand{\eg}[0]{\textit{e.g.}}
\newcommand{\etal}[0]{\textit{et al.}}

\newcommand{\keobf}{\textit{$(k,\epsilon)$-obfuscation}}
\newcommand{\argmin}{\operatornamewithlimits{argmin}}


\theoremstyle{plain}
\newtheorem{theorem}{Theorem} 
\newtheorem{definition}{Definition}
\newtheorem{problem}{Problem}
\newtheorem{example}{Example}
\newtheorem{lemma}{Lemma}

\newcommand{\methodName}{Squid}


%
\title{Sharing Uncertain Graphs with Syntactic Anonymity}

% \author{Dongqing Xiao,  Mohamed Y. Eltabakh, Xiangnan Kong \\ 
% \affaddr{Worcester Polytechnic Institute (WPI), Computer Science Department, MA, USA} \\
% \{dxiao,~meltabakh,~xkong\}~@wpi.edu}  
\author{%
% author names are typeset in 11pt, which is the default size in the author block
{Dongqing Xiao, Mohamed Y. Eltabakh, Xiangnan Kong}%
% add some space between author names and affils
\vspace{1.4mm}\\
\fontsize{10}{10}\selectfont\itshape
Computer Science Department, Worcester Polytechnic Institute \\
Worcester, United States of America\\
\fontsize{9}{9}\selectfont\ttfamily\upshape
\{dxiao,~meltabakh,~xkong\}@wpi.edu\
}
\begin{document}
\maketitle


% NOTE keywords are not used for conference papers so do not populate them
% \begin{keywords}
% keyword-1, keyword-2, keyword-3
% \end{keywords}
%

\begin{abstract}  
Many graphs in real-world social and business applications are not deterministic, but are uncertain in nature. 
Related research requires open access to such uncertain graph datasets. While sharing these datasets often risks exposing 
sensitive user data to the public. 
However, current privacy preserving graph releasing works mainly target on deterministic graphs and overlook the uncertain scenario. 

% We first show that simply combining the representative extraction strategy and conventional graph anonymization method will result in the addition of noise that significantly disrupts uncertain graph structure. 
We first show conventional methods tailored towards deterministic graphs are not applicable. 
They significantly disrupt the structure of uncertain graph because of the disregarding the possible world semantic of uncertain graphs. 
Our work seeks a solution to release uncertain graphs with high utility without compromising user privacy. We introduce an uncertainty-aware method that provides enough privacy guarantee with much less utility loss. With the possible world semantics, it enables a fine-grained control over the injected noise. Finally, we apply our method to real uncertain graphs and show that it produces sanitized graphs that closely match the original uncertain graphs. 
% not only the statstics but also real-world applications 
\end{abstract}

\section{Introduction}

\label{sec:Intro}

% Graph-- Uncertain Graph--Examples 
Graphs are used to capture the complex relationships in emerging applications, such as business to business (B2B) and social networks. 
Sometimes, the existence of the relationship between two entities is uncertain. For instance, in social networks, nodes represents individual users, while edges represent friendship or trust link among them.  Usually, the link is derived by inference and prediction models built on interaction details~\cite{Lin_B2B,Adar_Managing_2007,Kempe_Maximizing_2003}. And, edge probability denotes the accuracy of a link prediction, or the trust of one person on another. 
In these applications, the data can be modeled and shared as uncertain graphs whose edges carries a probability of existence. The probability represents the confidence that the relationship holds in reality. 

\begin{figure}[!htb]
  \vspace{-7pt}
    \subfigure[Social Trust Network]{\label{fig:socialNetwork}
      \begin{minipage}[l]{0.46\columnwidth}
        \centering
        \includegraphics[height=2.7cm]{ill/SocialNetwork.pdf}
      \end{minipage}
      }
    \subfigure[B2B Network]{\label{fig:b2bNetwork}
      \begin{minipage}[l]{0.46\columnwidth}
        \centering
        \includegraphics[height=2.7cm]{ill/B2BNetwork.pdf}
      \end{minipage}
      }
    \vspace{-7pt}
    \caption{Real-world uncertain graphs with privacy concerns.}
    \label{fig:motivation}
    \vspace{-7pt}
\end{figure} 

% Sharing--Privacy--Examples 
These uncertain graphs are invaluable for scientific research and commercial applications~\cite{Kempe_Maximizing_2003,Cho_Friendship_2011}. However, sharing these uncertain graphs would violate the privacy of users or entities profiled inside. In social trust network, the trust relationships among users--- which greatly impact users' behaviors, are usually probabilistic.  They are useful in social interaction study and micro-targeting. While users are unwilling to share such confidential information with potential adversaries like Cambridge Analytica. In B2B networks, business operators also hesitate to share transaction patterns as it relates to  confidential business models. Such tension is raising the question of sharing uncertain graphs without compromising privacy. 


% State-of-Art 
A number of privacy preserving graph sharing schemes have been studied in the deterministic scenario~\cite{Liu_Towards_2008,Ying_Randomizing_2008,Wang2011,Liu_Privacy_2009,Nguyen_Anonymizing_2015,Sala_Sharing_2011,Xiao_Differentially_2014,lee2011}, though many problems about sharing uncertain graphs still remain unexplored.

%weight casting fails 
An obvious approach is to convert uncertain graph sharing problem into the deterministic case by casting edge probabilities as edge weights. However, by disregarding the possible world semantics of the uncertain graph, such an approach fails to reflect uncertain graph properties such as connectivity, dense subgraphs correctly~\cite{Zhao_Detecting_2014,Hua_Probabilistic_2010}. {\small Note that connectivity of deterministic subgraphs is generally measured by the concept of cut, which is defined as the sum of weights of intra edges. Generally, the bigger the cut, the harder to separate two subgraphs. In Figure~\ref{fig:socialNetwork}, the equal cut $C(SG_{1},SG_{2})=C(SG_{3},SG_{2})=1$ implies the identical connectivity of $SG_{1}$ and $SG_{3}$ w.r.t $SG_{2}$. However, with the possible world semantics, we know the probability to separate $SG_{1}$ and $SG_{2}$ is $(1-0.5)^{2}=0.25$, and that to separate $SG_{2}$ and $SG_{3}$ is $(1-0.3)(1-0.7)=0.21$. Hence, in fact, $SG_{2}$ is closer to $SG_{1}$ than to $SG_{3}$.} 
Hence, the weighted graph anonymization scheme could produce very poor result in the uncertain scenario even if the anonymization algorithm is good. 


% Rep-OB 
In previous work~\cite{Xiao:2018}, we ever present another option, Rep-An. 
It first extracts a single deterministic representative
instance $G$ that capture structural properties of the uncertain
graph.
After that, anonymization can be then be proceed efficiently on $G$ using conventional algorithms, regardless of the uncertainty.  
However, Rep-An is is not always feasible. The detachment of edge uncertainty deteriorates the data utility. 


As ever mentioned, either the casting or detachment of edge probabilities lead to poor results. Existing graph anonymization schemes are inadequate to share uncertain graphs with desirable trade-off between privacy and utility. 
There are the following distinct challenges in the uncertain scenario. 

$\bullet$~\textup{\emph{Stochastic Privacy Attacks.}}~~Edge uncertainty plays an indispensable role in the uncertain graph model. It is impractical to discard them in the release.  
While the extra release of edge uncertainty makes privacy protection far more difficult as it empowers the adversary and makes the profiled entity more vulnerable. 
% To this end, we show the potential re-identification attack and present the corresponding solution. 

$\bullet$~\textup{\emph{Stochastic Utility Loss Metric.}}~~It is very challenging to maintain the graph structure when the uncertain graph is modified to pursue anonymity. The structural distortion incurred is evaluated by the corresponding metric.  It plays the key role in utility preserving. Unfortunately, existing graph utility loss metrics such as graph edit distance~\cite{Liu_Towards_2008}, spectrum discrepancy~\cite{Ying_Randomizing_2008}, community reconstruction error~\cite{Wang2011} and shortest path discrepancy~\cite{Liu_Privacy_2009} are not suitable in the uncertain scenario because of the ignorance of edge probability. 
% In this context, the discrepancy w.r.t standard uncertain graph reliability becomes a good criterion. It evaluates the connectivity difference in the context of the entire graph and meanwhile utilizes the possible world model. 

$\bullet$~\textup{\emph{Intractable Search Space.}}~~It is very challenging to find a sanitized graph with the desired level of privacy by as few graph operations as possible. 
In the deterministic scenario, the problem is known to be NP-hard by edge additions and deletions~\cite{Hartung_Theory_2015}. 
In the uncertain scenario, the edge modification is no longer a binary operation (addition/deletion), but can be infinite probability values. Exhaustive search is computationally intractable if the number of edges is large.
 % Thus, we approximate the problem of interest via a randomized algorithm, which built on the basis of meta-heuristics. It excels in identifying a population of sanitized results with good quality.

In this work, we propose a solution tailored towards uncertain graphs via incorporating possible world semantics, {\methodName}. 
% {\methodName} is built on the basis of syntactic private notation. 
It preserves as much the stochastic nature of the original uncertain graph as possible, while injecting enough structural noise to guarantee a chosen level of privacy against re-identification attacks.
Specifically, we make the following contributions:
\begin{itemize}
\item To the best of our knowledge, we are the first to formulate the uncertain graph anonymization problem. 
 We show the potential re-identification attack and present the corresponding privacy notion. 
\item We propose an utility loss metric on the basis of generalized reliability. It evaluates the connectivity difference in the context of the entire graph and meanwhile utilizes the possible world model. 
\item We propose a randomized algorithm with dual meta-heuristics. By incorporating the possible world semantics, it excels in identifying a population of sanitized results with good quality.
\item We conduct extensive experimental studies to demonstrate efficiency and practical utility of our algorithms.
\end{itemize}

The rest of the paper is organized as follows. In Section 2, we summarize related works and clarify our distinct privacy goal. In Section 3 we formulate the uncertain graph-anonymization problem. Sections 4 – 5 consider the anonymization problem in the context of uncertain graphs.  In Section 6 we apply our method to several real-world uncertain graphs, and demonstrate their efficiency and practical utility. 


\section{Related Works}
A significant amount of prior work has been done protecting the privacy of network datasets.
We summarize them here and clarify our privacy goals in this paper. 

\textbf{Syntactic Privacy.}~~Early works on privacy-preserving network publishing mainly focus on developing anonymization techniques for deterministic graphs. They are designed to \emph{publish} the data in an anonymized manner without making any assumptions of the type of analysis and queries that will be executed on the release one. Once the data is published, it is available for any type of analysis. Most of them leverage \emph{syntactic} privacy models derived from $k$-anonymity~\cite{Sweeney:2002:KAM:774544.774552} to create $k$ identical neighborhoods, or $k$ identical degree nodes. Following this path, many graph anonymization techniques have been proposed.

\textbf{Graph Anonymization Techniques.}~~Existing methods for anonymizing ``graphs" can be classified into four main categories: (1) Clustering-based generalization~\cite{Hay_Anonymizing_2007,Bhagat_Class_2009,hay2010resisting}; (2)~{\em Edge modification}~\cite{Liu_Towards_2008, Zhou_Preserving_2008, Wang2011, Wu_k_2010, Skarkala_Privacy_2012}, 
(3)~{\em Edge randomization}~\cite{Liu_Privacy_2009,Ying_Randomizing_2008, Ninggal_Utility_2015},
and~(4)~{\em Uncertainty semantic-based modifications} which add uncertainty to some edges and thus converting the graph to an uncertain version~\cite{Boldi_Injecting_2012, Nguyen_Anonymizing_2015}. The uncertainty semantic-based approaches transform the original deterministic graph into an uncertain one to be published. These techniques are known as the state-of-art ones because of their excellent privacy-utility tradeoff brought by the fine-grained perturbation leveraging the uncertain semantics. To the best of our knowledge, these techniques are tailored to \emph{deterministic} graphs (unweighted \& weighted) that overlook edge uncertainty.  

\textbf{Diffiential Privacy.}~~Another avenue is to apply differential privacy to provide privacy guarantee for network data. It roughly falls into two directions. The first direction aims to release certain differentially private \emph{data mining results}, such as degree distributions, sub-graph counts, and frequent graph patterns. Such methods that release only query results require tracking the results: early uses of the data can affect the quality of later uses, thus no new queries can be permitted on the data. For example, the purpose here is to let researchers develop new algorithms. It is unclear how this can be done without access to network data. The second direction aims to publish a sanitized graph. Most research in this direction projects an input graph to dK-series and ensures differential privacy on dK-series statistics. These private statistics are then either fed into generators or MCMC process to generate a fit Syntactic graphs. While current techniques are still inadequate to provide desirable data utility for many graph mining tasks. 

\textbf{Our Goal: Data Model \&~Privacy Policy.}~~
In this work, we study the problem of privacy preserving \emph{uncertain} graph publishing. Conceptually, the problem can be interpreted as a natural generalization of the \emph{determinitic} graph contexts to a larger probabilistic context, with the anonymization process being specifically optimized. 

Here, we remind the reader the approach that first casting the probability of every edge into a weight then applies existing anonymization methods on this weighted graph to attain the anonymized uncertain graph is problematic. First, there is no meaningful way to perform such casting. The casting has been proven to be erroneous in various uncertain graph mining tasks~\cite{Potamias_K_2010,Zhao_Detecting_2014}. Second, there is no principled way to additionally encode normal weights on the edge. For example, each link in the road network can be weighted indicating the distance or travel time between them, and a probability can be assigned to model the likelihood of a traffic jam~\cite{Jin_Distance_2011}. In summary, existing strategies for weighted graphs anonymization cannot be applied to \emph{uncertain} graphs.

In the context of privacy preserving graph publishing, we can choose to adopt the Syntactic or differential privacy policy.
$\epsilon$-differential privacy does relate to individual identifiability and provides strong privacy guarantee without making any assumption of privacy risks. However, there is no clear way to set a general policy for a value $\epsilon$ that provides sufficient privacy~\cite{lee2011}. In contrast to $\epsilon$-differential privacy, Syntactic privacy model can generally be defined and understood based on the data schema; parameters have a clear privacy meaning that can be understood independent of the actual data and have a clear relationship to the legal concept of individual identifiability of data. In this work, we choose $k$-obfuscation, a variant of $k$-anonymity as the basis of our privacy policy for uncertain graph \emph{publishing}.

% Add more sentence to uncertain graphs ... 
% Our general approach is to produce Syntactic graphs by adding controlled perturbations to the structure of the original \emph{uncertain graph}. The generic approach can provide protection against identity disclosure and link disclosure. The choice directly impacts how much structural noise must be introduced to obtain a given level of privacy guarantees. In this paper, we chose to target against node re-identification problem for {\emph uncertain graph} releasing.  
\section{Problem Definition}
\label{sec:notation}

In this section, we present the notations, definitions and the problem formulation.  
% In this section, we present the models of uncertain graph, privacy criteria, and utility metric. We then present the formal formulation of 
% uncertain graph anonymization problem. 

\subsection{Uncertain Graph}
% Let $\mathcal{G}=(V,E,\mathit{p})$ be an uncertain graph, where $V$ is the set of nodes, $E$ is the set of edges, and function $\mathit{p}: E \rightarrow [0,1]$ assigns a  probability of existence to each edge, denoted as $\mathit{p}(e)$. 
An uncertain graph $\mathcal{G}=(V,E,\mathit{p})$, is defined over a set of nodes $V$, a set of edges $E$, and a set of probabilities $\mathit{p}$ of edge existence. Following the literature, we consider the edge probabilities independent~\cite{Potamias_K_2010,Zhao_Detecting_2014,Hua_Probabilistic_2010,Jin_Distance_2011}, and we assume \emph{possible-worlds} semantics~\cite{Colbourn_Colbourn_1987}. Specifically, the \emph{possible world} semantics interprets $\mathcal{G}$ as a set of possible deterministic graphs 
$W(\mathcal{G}) = \{G_1, G_2, ..., G_n\}$, where each deterministic graph $G_i \in W(\mathcal{G})$ includes all vertices of $\mathcal{G}$ 
and a subset of edges $E_{G_i} \subset E$.  
The probability of observing any possible world $G_i=(V,E_{G_i}) \in W(\mathcal{G})$ is    
\begin{equation*}
    Pr[G_i]=\prod_{e \in E_{G_i}} {\mathit{p}(e)} \prod_{e \in E \setminus E_{G_i}} (1-\mathit{p}(e))
\end{equation*}

In this work, we assume the input uncertain graph undirected and contains no self-loops or multiple edges. 


\subsection{Reliability-Based Utility Loss Metric}
A well-chosen utility-loss metric may lead to substantially less sanitized graphs at a minimal loss of information. 
As be known to all, connectivity is a fundamental graph property and plays an important role in graph mining tasks such as locating $k-$nearest neighbor~\cite{Potamias_K_2010}, graph clustering~\cite{Asthana_Predicting_2004} and shortest paths detecting~\cite{Zhao_Detecting_2014}. The connectivity model has been shown to be able to yield better representation than degree sequence model. The connectivity discrepancy was proven to be a proper utility-loss metric. In this paper, we use its generalized version -- Reliability Discrepancy as the utility-loss metric in the uncertain graph context. 
 
In uncertain graphs, the concept of reliability is used to generalize \emph{connectivity} by  capturing the probability that two given (sets of) nodes are reachable over all possible worlds of the uncertain graph as follows:
\begin{definition}
    \textbf{Two-Terminal Reliability~\cite{Colbourn_Colbourn_1987}}  Given an uncertain graph $\mathcal{G}$, and two distinct nodes $u$ and $v$  $\in~V$, the reliability of $(u,v)$ is defined as:
        \begin{equation*}
                R_{u,v}(\mathcal{G})= \sum_{G \in W(\mathcal{G})}  \mathcal{I}_{G}(u,v) Pr[G] 
        \end{equation*}
    where $\mathcal{I}_{G}(u,v)$ is 1 iff $u$ and $v$ are contained in a connected component in $G$, and 0 otherwise.   
    \label{d:reliability}
\end{definition}


\theoremstyle{definition}
\begin{definition}
    \textbf{Graph Reliability Discrepancy}
    The reliability discrepancy of graph $\tilde{\mathcal{G}}=(V,E, \tilde{\mathit{p}})$, 
    denoted as $\Delta(\tilde{\mathcal{G}})$, 
    w.r.t. an original graph  $\mathcal{G}=(V,E,\mathit{p})$  is 
    defined as the sum of the two-terminal reliability discrepancy over all node pair $(u,v) \in V_\mathcal{G}$.
    \begin{equation*}
        \Delta(\tilde{\mathcal{G}})=\sum_{(u,v) \in V_\mathcal{G} }|R_{u,v}(\mathcal{G})-R_{u,v}(\tilde{\mathcal{G}})|
    \end{equation*}
\end{definition}


\subsection{Attack Model and Privacy Criteria}
\label{sec:AMPC}
In this paper, we focus on the ``identity disclosure problem"~\cite{Liu_Towards_2008} over uncertain graphs, which is one serious privacy leak concern when a graph dataset is published. Formally, give a published graph $G$, if and adversary can locate the target entity $t$ as a vertex $v$ of $G$ with a high probability via auxiliary information, we said that the identity of $t$ is disclosed. The popular assumption of auxiliary information is node degree~\cite{Liu_Towards_2008}. 

Following the literature, we adopt the syntactic $\keobf$ criterion~\cite{Boldi_Injecting_2012} for privacy guarantee. 
Analogous to the well known $k$-anonymity notion, $k$-obf requires to blend every vertex with \emph{other} fuzzy-matching nodes. 
Compared to $k$-anonymity, $k$-obf, which is global and entropy-based quantification, is more adequate than the previous used local quantification based on a posteriori belief probabilities. An excellent discussion on $k$-obf was presented by Bonchi {\etal}~\cite{Bonchi_Identity_2014}. 
Moreover, the introduction of a tolerance parameter $\epsilon$, which allows skipping up to $\epsilon * |V|$ nodes, makes it more practical. The skipped nodes might be extreme unique nodes, e.g., Trump in a Twitter network, whose obfuscation is almost impossible.
The formal definition is as follows:
\theoremstyle{definition}
\begin{definition}
	\textbf{\boldmath{$(k,\epsilon)$}-obf \cite{Boldi_Injecting_2012}}
    Let $P$ be a vertex property (i.e., vertex degree in our work), $k \geq 1$ be a desired level of anonymity, and $\epsilon >0 $ be a tolerance parameter. 
    An sanitized uncertain graph $\tilde{\mathcal{G}}$ is said to $k$-obfuscate a given vertex $v \in \mathcal{G}$ w.r.t $P$ if the entropy $H()$ of the distribution $Y_{P(v)}$ over the nodes 
    of $\tilde{\mathcal{G}}$ is greater than or equals to $\log_{2}{k}$:
    \begin{equation*}
        H(Y_{P(v)}) \geq \log_{2}{k}.
    \label{obfCon}
    \end{equation*}
The uncertain graph $\tilde{\mathcal{G}}$
is $(k,\epsilon)$-obf w.r.t property $P$ 
if it $k$-obfuscates at least $(1-\epsilon)|V|$ nodes in $\mathcal{G}$. 
\label{def:obf}
\end{definition} 

\subsection{Problem Statement}
% add one sentence to address this one 
Given the above foundation, we can now formulate the addressed problem.  
\begin{problem}
	\textbf{Reliability-Preserving Uncertain Graph Anonymization}
     Given an uncertain graph $\mathcal{G}=(V,E,\mathit{p})$ and anonymization parameters $k$ and $\epsilon$, 
     the objective is to find a  $(k,\epsilon)$-obfuscated uncertain graph $\tilde{\mathcal{G}}=(V,E,\tilde{\mathit{p}})$ 
     with minimal  $\Delta(\tilde{\mathcal{G}})$. That is:
     \begin{equation*}
             \begin{aligned}
                 & \argmin_{\tilde{
                \mathcal{G}}} & & \Delta(\tilde{\mathcal{G}}) \\
                &  \text{Subject to} & &\tilde{\mathcal{G}} \text{~is~} (k,\epsilon)-obf
            \end{aligned}
     \end{equation*}
     \label{prob:unobf}
\end{problem}
% \section{The Representative Anonymization Algorithm}
\label{sec:repOB}

We now introduce the representative anonymization algorithm (Rep-An) that combines isolated but complementary work from literature for uncertain graph anonymization. As shown in Figure\ref{fig:repOB}, we first extract a single \emph{representative} instance from an original uncertain graph. Then, conventional anonymization techniques can be then applied on this representative to attain closely approximate anonymized output of the original uncertain graph. There has been extensive work on extracting a single representative instance of uncertain graphs that capturing graph statistics such as the expected vertex degrees~\cite{Parchas_Gullo_Papadias_Bonchi_2014}. This body of research comes to its aid that anonymization can be carried out on uncertain graphs without special design for edge uncertainty.

However, the approach has several limitations. First, the input edge uncertainties (probabilities) are no longer integrated into the anonymization process since they are detached from the graph in the first phase. Second, the anonymization process (the second phase) is oblivious to the {\em reliability} metric since its input is a made-up deterministic graph. Third, since the two phases are isolated from each other, different phases are optimized for different metrics. As the result, this naive \texttt{Rep-An} approach introduces a high level of noise and consequently deteriorates the overall utility of the anonymized graph. 
In the experiment section, we further study this approach empirically and confirm its impracticality.

% Figure~\ref{fig:repOB} illustrates these limitations. The uncertain input graph (L.H.S) will have 
% the corresponding deterministic representative graph (middle) according to~\cite{Parchas_Gullo_Papadias_Bonchi_2014}. 
% This graph is viewed by the state-of-art anonymization techniques as being already anonymized and will be published as is (R.H.S top graph).
% However, it is clear that an anonymized graph with a much-higher utility can be generated, e.g., (R.H.S bottom graph).  

% % {\todo{Still References in the figure need to be fixed...}}
% % the utility loss between XXX and XXX is wrong expression 
% % {\todo{Remove the 2nd half of that Figure 3. I does not add much... And it needs more discussion on why the top part generates edges of weight 1, which does not seem to be uncertain, etc...}} 
% % {\todoB{References in the figure (on the edges) do not match the ones in text!!! Update to make them matching...}}

\begin{figure}[t]
	\vspace{-1em}
    \captionsetup{margin=0cm}
    \centering  
        \includegraphics[width=0.95\columnwidth]{AddFigure/repOB.pdf}
        \vspace{-0.7em}
    	\caption{Overview of Rep-An. Noise is added to the extracted \emph{representative} instance.}
    \label{fig:repOB}
    \vspace{-0.5em}
\end{figure}

\section{Privacy Via XXX}
\label{sec:tech}
Instead of detaching edge uncertainty from the anonymization phase, we shift the state-of-art method by integrating uncertainty semantics into its core steps, namely XXX.
It enables a unifying and grained control over the noise injected to uncertain graphs, then provides enough privacy guarantee with good utility.

\subsection{The State-of-Art Framework}
\textbf{Problem Transformation}~~Anonymization is done via altering the probabilities of sampled edges. For each sampled edge $e$, it is assigned a probability deviation $r_{e}$, where $r_{e} \leftarrow R(\sigma)$. As shown in the work~\cite{Boldi_Injecting_2012}, the distribution $R(\sigma)$ is a truncated normal distribution with mean 0 and variance $\sigma^2$. But, it could in principle be any distribution. 

As the standard deviation $\sigma$ decreases, a greater mass of $R_{\sigma}$ will concentrate near $r_{e}=0$, then the amount of injected noise and consequent structural deviation will be smaller. It enables transforming the graph anonymization problem into the minimization of structural noise need. The later one can be achieved via a binary search on the value of standard deviation $\sigma$.
\begin{algorithm}
% {\scriptsize
	\begin{algorithmic}[1]
    	\item[] {\textbf{Input:}~Uncertain graph $\mathcal{G}$, adversary knowledge $\mathcal{K}$, obfuscation level $k$, tolerance level $\epsilon$, size multiplier $c$ and white noise level $q$ }
        \item[] {\textbf{Output:}~The anonymized result $\tilde{\mathcal{G}}_{obf}$}
     	\STATE {$\sigma_{l} \leftarrow 0$; $\sigma_{u} \leftarrow 1$} \\
        \REPEAT
        \STATE{$\langle \tilde{\epsilon}, \tilde{\mathcal{G}} \rangle$ $\leftarrow$ \texttt{GenObf}($\mathcal{G},k,\epsilon,c,q,\sigma_{u}, \mathcal{K}$)} \\
        \STATE{{\bf if} $\tilde{\epsilon}=1$ (fail) {\bf then} $\sigma_{l} \leftarrow \sigma_{u}$; $\sigma_{u} \leftarrow 2\sigma_{u}$}
        \UNTIL{$\tilde{\epsilon} \neq 1 $} \\
        \REPEAT
        	\STATE {$\sigma \leftarrow (\sigma_{u}+\sigma_{l})/2$}
            \STATE {$\langle \tilde{\epsilon}, \tilde{\mathcal{G}} \rangle \leftarrow \texttt{GenObf}(\mathcal{G},k,\epsilon,c,q,\sigma_{u}, \mathcal{K})$}
            \STATE {{\bf if} $\tilde{\epsilon} =1$~{\bf then}~$\sigma_{l} \leftarrow \sigma$}
            \STATE {{\bf else} $\sigma_{u} \leftarrow \sigma;~~\tilde{\mathcal{G}}_{obf} \leftarrow \tilde{\mathcal{G}}$}
        \UNTIL{$\sigma_{u}-\sigma_{l}$ is enough small}
        % \COMMENT{\textcolor{blue}{\scriptsize Binary search for better obfuscation}}
        \STATE {return $\tilde{\mathcal{G}}_{obf}$}
    	\caption{\SysName Iterative Skeleton}
	 \label{alg:Skeleton}
    \end{algorithmic}
    % }
\end{algorithm}
\vspace{-5pt}


\textbf{Search Flow}~~The binary search flow is determined by the \textmd{GenObf} function. For a given value of standard deviation $\sigma$, \textmd{GenObf} either returns the best found {\keobf} instance or indicates failure. 
The binary-search process starts with an initial guess of an upper bound $\sigma_{u}$ which is iteratively doubled until a ${\keobf}$ graph is found. Then, it is performed using $\sigma_{l}$ = 0 as the lower bound, and the found upper bound $\sigma_{u}$. The binary search terminates when the search interval is sufficiently short. It outputs the best {\keobf} found (i.e., the
 last one that was successfully generated).




% \section{Experiments}
\label{sec:exp}
\begin{figure*}[!htb]
    \centering
    \vspace{-5pt}
    \includegraphics[width=\linewidth]{exp/exp_rel.jpg}
    \caption{Comparison of anonymization methods in terms of their ability to preserve Reliability.}
    \label{fig:ex_rel}
    \vspace{-5pt}
\end{figure*}
\begin{figure*}[!htb]
    \vspace{-5pt}
    \centering
    \includegraphics[width=\linewidth]{exp/exp_degree.jpg}
    \caption{Comparison of anonymization methods in terms of their ability to preserve Average Node Degree.}
    \label{fig:ex_degree}
    \vspace{-5pt}
\end{figure*}

In this section, we evaluate how well Chameleon and its variants preserve a \emph{uncertain} graph's structural statistics by comparing its anonymized \emph{uncertain} graphs against the original one in terms of graph metrics. The evaluated methods are Chameleon (RSME), RS (Reliability Sensitive), ME (Maximization Entropy) and Rep-An (Representative Anonymization).
Strong structural similarity (small error) in these results would establish the utility of these anonymized graphs in real research analysis and experiments. 

\subsection{Evaluation Methodology}

%-----------------------------------
%  Compared methods
\begin{table}[t]
    \small
    \caption{Summary of compared methods. }
    % \vspace{-1em}
    \newcommand{\tabincell}[2]{\begin{tabular}{@{}#1@{}}#2\end{tabular}}
    \centering
    \scalebox{0.95}{
    \begin{tabular} {|l|c c c|c|}
    \hline 
    Method & \tabincell{c}{Uncertainty \\-aware }  &  \tabincell{c}{Reliability \\-oriented} & \tabincell{c}{Anonymity \\-oriented} & Source \\
    \hline   \noalign{\vskip 1.5mm} \hline
    Rep-An  & -- & --& \checkmark & \cite{Parchas_Gullo_Papadias_Bonchi_2014}+\cite{Boldi_Injecting_2012} \\
    \hline 
    RSME & \checkmark & \checkmark & \checkmark & This work \\
    \hline 
    ME & \checkmark  & -- & \checkmark &  This work   \\
    \hline 
    RS & \checkmark  &\checkmark & --  &   This work  \\
    \hline 

   \end{tabular}
   }
   \vspace{-10pt}
    \label{tab:methods}
\end{table}
%-----------------------------------

\textbf{Metrics.}~~Besides reliability, our evaluation includes three classes of graph metrics. One group includes degree-based metrics such as Average Node Degree, Degree Distribution, Maximal Degree. These are basic topological metrics capture how degree distributed among nodes. The second group includes node separation metrics such as Average Distance, Graph Diameter. They are used to quantifying the inter-connectivity of the graph. The third group metrics include Clustering Coefficient which measures how close neighbors of a node are to forming cliques. 

\textbf{Computation.}~~Since there does not exist the closed formula for graph metrics expect Average Node Degree, the results are approximated by Monte Carlo sampling. Specifically, we create a number of random instances of an uncertain graph, and we compute the expected value of each metric using the average of the sampled graphs. Here, we use 1,000 samples since it has been shown that $1000$ usually suffices to achieve accuracy converge~\cite{Potamias_K_2010,Jin_Distance_2011}. In particular, we use Approximate Neighborhood Function (ANF) \cite{Boldi_Rosa_Vigna_2011}, to approximate shortest path-based statistics. For each metric, we report the ratio of absolute difference against the original one. 

\textbf{Parameter Setting.}~~We generate anonymized \emph{uncertain} graphs for $k \in [100,300]$ and compare the graph metrics of the resulting \emph{uncertain} graphs against those original graphs. We limit ourselves to obfuscation levels, $k \in [100,300]$ following reasons. 
First, we aim to explore the case the desired privacy level requires a small amount of noise ($k=100$). This way, we can quantify the utility loss difference introduced by Chameleon against Rep-An method. Second, it naturally requires a high level of noise to provide strong levels of privacy guarantees. We want to explore the sensitivity of different variants. Unfortunately, very large values of $k$ require large noise, thus producing anonymized graphs that are extremely different from the original ($k=300$). 


\subsection{Results}
\textbf{Reliability.}~~Figure~\ref{fig:ex_rel} compares the average reliability discrepancies. The smaller discrepancy, the better the reliability preserving. In all cases, the Chameleon (RSME) performs best, followed by its variants (RS and ME).

For each of the DBLP, BRIGHTKITE and PPI graphs, the reliability of Chameleon ($k=100$) output graphs are very close the ones of the original graphs. When we increase the strength of the privacy guarantees, i.e., larger $k$ values of $200$ and $300$, the error of average reliability progressively increases. Our experiments show the errors introduced by Chameleon (RSME) only increase slowly as $k$ value increases. Note that RS strategy (less perturbation over ``bridge"-like edges) does provide the beneficial impact on the reliability preserving (as witnessed by the smaller error introduced by RSME v.s. ME).

\textbf{Degree-based Metrics.}~~For brevity, we only report results for  Average Node Degree, as representative of degree-based metrics in Figure~\ref{fig:ex_degree}. For each of the DBLP, BRIGHTKITE and PPI graphs, the average node degrees of Chameleon ($k=100$) output graphs are very close the ones of the original graphs. When we increase the strength of the privacy guarantees, i.e., larger $k$ values of $200$ and $300$, the error of average degree slowly increases. For example, DBLP graph shows a small deviation even for $k=300$. The worst-case average degree deviation introduced by the RSME method is still within 15\% of the original. 

On the other hand, BRIGHTKITE and PPI show slightly different behaviors. For large $k$ values, i.e., $k=200$ and $k=300$, the largest error is over 300\% from the original values. The root of large errors is the existence of heavy-tailed degree distribution (over 4\%) in these two graphs which requires larger amount of noise. 

Among Chameleon variants, RS is clearly outperformed by ME and RSME methods in all datasets. For example.  average degree deviation introduced by RS (1.16) is much larger ones of other variants (0.39 and 0.27) in DBLP dataset. This is encouraging because we can eliminate the bulk of the error by fine-grained control of edge probability alteration.   

\begin{figure*}[!tb]
    \centering
    \vspace{-5pt}
    \includegraphics[width=\linewidth]{exp/exp_apd.jpg}
    \caption{Comparison of anonymization methods in terms of their ability to preserve Average Distance.}
    \label{fig:ex_apd}
    \vspace{-5pt}
\end{figure*}

\begin{figure*}[!tb]
    \centering
    \vspace{-5pt}
    \includegraphics[width=\linewidth]{exp/exp_cc.jpg}
    \caption{Comparison of anonymization methods in terms of their ability to preserve Clustering Coefficient.}
    \vspace{-5pt}
    \label{fig:ex_cc}
\end{figure*}
% \vspace{-5pt}

\textbf{Node Separation Metrics.}~~For brevity, we report only the Average Distance as a representative of the node separation metrics. Figure~\ref{fig:ex_apd} shows the Average Distance (AD) values computed on DBLP, BRIGHTKITE, and PPI graphs compared to the AD values on their anonymized graphs.  

\textbf{Clustering Coefficient}~~In comparison, the error introduced by strengthening privacy (i.e., increasing $k$) is relatively small. As with previous experiments, the BRIGHTKITE graph shows a little different behavior. In this case, all of Chameleon output graphs do a good job of preserving Clustering Coefficient of the original graphs (well below 15\%). It confirms the close relationship between reliability and clustering coefficient.


\textbf{Summary.}~~Our experimental evaluation on real-world datasets confirms the initial and driving intuition: the {\SysNameNS} approach which explicitly incorporates edge uncertainty and the possible world semantic in the anonymization process outperforms the benchmark solution Rep-An significantly regarding the uncertain graph utility preservation. The {\SysNameNS} introduces limited impact as a result of adding noise to guarantee privacy.  

Another take-home message is: by using fine-grained and uncertainty-aware perturbation strategies such as reliability sensitive edge selection (RS) and max entropy based edge prob alteration (ME), one can achieve the same desired level of obfuscation with the smaller change on the uncertain graph thus maintaining higher data utility. 





% \section{Conclusion}
In this work, we first identify the overlooked problem–uncertain graph anonymization. 
We develop a new scheme, Squid, which integrates edge uncertainty into the core. 
It excels in identifying sanitized uncertain graphs with excellent quality. 
Experiments on three real-world datasets verify its effectiveness. 
In real-world graphs, edge probabilities sometimes are not independent, but dependent.  We leave the conditional probability model as a future extension. 
Another extension is to investigate sharing uncertain graphs in the differentially private manner.
\bibliographystyle{abbrvnat}
% \bibliographystyle{abbrv}
\bibliography{refs}

\end{document}
