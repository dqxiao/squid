\section{Conclusion}
In this work, we study the problem of developing a flexible \emph{uncertain} graph anonymization scheme that preserves graph structure while providing the user-specified level of privacy guarantees. First, we introduce the Rep-An scheme that aims these goals using the single \emph{representative} instance as an approximation of \emph{uncertain} graph and show this scheme requires the addition of high levels of noise to obtain privacy guarantee. 
Second, we develop the {\SysNameNS} approach that seamlessly integrates edge uncertainty into the core of the
anonymization process such as the evaluation of privacy risk, utility loss, and judicious \emph{uncertain} graph modifications. 
In particular, we present a new utility-loss metric based on the solid connectivity-based graph model under the possible world semantics, namely the reliability discrepancy. Moreover, we introduce the reliability-sensitive edge selection strategy (RS) and max-entropy edge perturbation (ME) strategy for fine-grained control of noise injecting. 
Experimental studies on different real-world datasets demonstrate that our approach can anonymize uncertain graphs to the desired anonymity level with at a slight cost of utility. 